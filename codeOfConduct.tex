\documentclass[12pt]{article}

\usepackage{dsfont}
\usepackage{enumerate}
\usepackage{fullpage}
\usepackage{parskip}

\setlength{\parindent}{0mm}

\title{codeOfConduct}
\author{simcconnell }
\date{May 2019}

\begin{document}

\begin{center}
Code of Conduct
\\
Kylerec 2019
\end{center}

\bigskip

One of the primary goals of this workshop is to provide a supportive environment for personal and mathematical growth. We certainly don't expect misconduct from our participants; rather, the goal of this document is to encourage positive behaviors and attitudes which we hope participants will bring into the wider math community. In particular, we believe that clearly articulating our values is an important step toward building a space for collaboration where all individuals feel welcome. 

Exclusionary behavior and harassment will not be tolerated. We are committed to providing a safe and affirming environment for people of all races, gender identities and expressions, sexual orientations, physical abilities, ages, socioeconomic backgrounds, academic affiliations, and beliefs. 

The participants in Kylerec come from many different backgrounds, so we ask that you be mindful of impact your words will have on those around you. Remember that mindfulness is an active skill; it's possible to be hurtful without any sort of malicious intent. In particular, we're all still learning, so please try to be sensitive when pointing out mistakes. 

During talks, please be respectful. We're all eager to share our knowledge, but sometimes this enthusiasm can inadvertently lead to bad behavior. We'd like to help participants become more confident about giving math talks. One important aspect of this is responding to questions, so we ask that audience members give the speaker a chance to answer before jumping in. In addition, we'd like to include everyone in the discussion, which is only possible if we refrain from starting side conversations during talks. 

If you have any questions or concerns, please talk to an organizer. If you don't feel comfortable approaching someone directly, we're here to help resolve any issues. On the other hand, if you are told that you have (consciously or otherwise) made someone feel unwelcome, please try to respond in a positive way. Despite our best intentions, we all sometimes fail to live up to these standards. We don't expect perfection--- what matters is that we own up to our mistakes and make a consistent effort to improve. 

\end{document}
